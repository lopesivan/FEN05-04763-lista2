\documentclass{article}
\usepackage[utf8]{inputenc}

\title{%
  Tópicos Especiais em Eletrônica\\
  \large Lista 2\\
    Resolução}

\author{Ivan Carlos}
\date{October 2021}

\usepackage{natbib}
\usepackage{amsmath}  % matemática
\usepackage{amssymb}

\usepackage{blindtext} % links
\usepackage{hyperref}  % links

\usepackage{graphicx}
\usepackage{geometry}

\usepackage[utf8]{inputenc}
\usepackage[T1]{fontenc}
\usepackage[brazil]{babel}

\usepackage{color}    % suporte a cor
\usepackage{xcolor}
\usepackage{listings}
\usepackage{courier}
\lstset{
	basicstyle=\footnotesize\ttfamily, % Standardschrift
	%numbers=left,                % Ort der Zeilennummern
	numberstyle=\tiny,            % Stil der Zeilennummern
	%stepnumber=2,                % Abstand zwischen den Zeilennummern
	numbersep=5pt,                % Abstand der Nummern zum Text
	tabsize=2,                    % Groesse von Tabs
	extendedchars=true,           %
	breaklines=true,              % Zeilen werden Umgebrochen
	keywordstyle=\color{gray},
	%frame=b,
	%        keywordstyle=[1]\textbf,    % Stil der Keywords
	%        keywordstyle=[2]\textbf,    %
	%        keywordstyle=[3]\textbf,    %
	%        keywordstyle=[4]\textbf,   \sqrt{\sqrt{}} %
	showspaces=false,           % Leerzeichen anzeigen ?
	showtabs=false,             % Tabs anzeigen ?
	xleftmargin=17pt,
	framexleftmargin=17pt,
	framexrightmargin=5pt,
	framexbottommargin=4pt,
	%backgroundcolor=\color{lightgray},
	showstringspaces=false   % Leerzeichen in Strings anzeigen ?
	literate=
	{á}{{\'a}}1 {é}{{\'e}}1 {í}{{\'i}}1 {ó}{{\'o}}1 {ú}{{\'u}}1
	{Á}{{\'A}}1 {É}{{\'E}}1 {Í}{{\'I}}1 {Ó}{{\'O}}1 {Ú}{{\'U}}1
	{à}{{\`a}}1 {è}{{\`e}}1 {ì}{{\`i}}1 {ò}{{\`o}}1 {ù}{{\`u}}1
	{À}{{\`A}}1 {È}{{\'E}}1 {Ì}{{\`I}}1 {Ò}{{\`O}}1 {Ù}{{\`U}}1
	{ä}{{\"a}}1 {ë}{{\"e}}1 {ï}{{\"i}}1 {ö}{{\"o}}1 {ü}{{\"u}}1
	{Ä}{{\"A}}1 {Ë}{{\"E}}1 {Ï}{{\"I}}1 {Ö}{{\"O}}1 {Ü}{{\"U}}1
	{â}{{\^a}}1 {ê}{{\^e}}1 {î}{{\^i}}1 {ô}{{\^o}}1 {û}{{\^u}}1
	{Â}{{\^A}}1 {Ê}{{\^E}}1 {Î}{{\^I}}1 {Ô}{{\^O}}1 {Û}{{\^U}}1
	{ã}{{\~a}}1 {ẽ}{{\~e}}1 {ĩ}{{\~i}}1 {õ}{{\~o}}1 {ũ}{{\~u}}1
	{Ã}{{\~A}}1 {Ẽ}{{\~E}}1 {Ĩ}{{\~I}}1 {Õ}{{\~O}}1 {Ũ}{{\~U}}1
	{œ}{{\oe}}1 {Œ}{{\OE}}1 {æ}{{\ae}}1 {Æ}{{\AE}}1 {ß}{{\ss}}1
	{ű}{{\H{u}}}1 {Ű}{{\H{U}}}1 {ő}{{\H{o}}}1 {Ő}{{\H{O}}}1
	{ç}{{\c c}}1 {Ç}{{\c C}}1 {ø}{{\o}}1 {å}{{\r a}}1 {Å}{{\r A}}1
	{€}{{\euro}}1 {£}{{\pounds}}1 {«}{{\guillemotleft}}1
	{»}{{\guillemotright}}1 {ñ}{{\~n}}1 {Ñ}{{\~N}}1 {¿}{{?`}}1 {¡}{{!`}}1
}

\usepackage{caption}
\input{etc/color.tex}

%\usepackage{auto-pst-pdf} % ps

\usepackage{epstopdf}
\usepackage{pstricks}
\usepackage{pst-plot}
\usepackage{pstricks-add}
\usepackage{epsf}

\usepackage{graphicx}
\usepackage{subcaption}
%\pagestyle{empty}

\usepackage{floatrow}% images

\usepackage[numbered,framed]{matlab-prettifier}

%\geometry{left=2.5cm,right=2.5cm,top=2.5cm,bottom=2.5cm}

% mudando o idioma em listing -> Listagem
\renewcommand{\lstlistingname}{Algorítmo}
%\renewcommand{\lstlistingname}{Algorithm}
%\renewcommand{\lstlistlistingname}{Lista de Códigos Fonte}

% desabilita seções numeradas
\setcounter{secnumdepth}{0} % Do not enumerate a section

\begin{document}

\maketitle

\section{a)}

O primeiro passo é escrever nossa função abaixo.

\begin{equation} \label{eq1}
    F6(x,y) = 0.5-\frac{(\sin\sqrt{x^2+y^2})^2 -0.5}{(1+(0.0001)\cdot (x^2+y^2)))^2}
\end{equation}

\begin{figure}[h!]
\centering
\includegraphics[scale=.3]{images/fun.png}
\caption{Modelo da função F6}
\label{fig:func_f6_model}
\end{figure}

\begin{flushleft}
Dentro do diretório de trabalho criamos o arquivo ``F6.m'',
conforme mostrado abaixo.
\end{flushleft}

\lstinputlisting[style=Matlab-editor,label={F6},caption={Arquivo F6.m}, language={Matlab}]{code/F6.m}
\begin{flushleft}
Para invocar a função basicamente digitamos \emph{F6} munido de
suas entradas.

Dessa forma testamos as entradas $X=0, Y=0$ mencionadas no item
e verificamos se a saída é $Z=1$.
\end{flushleft}

\lstinputlisting[style=Matlab-editor,label={prompt1},caption={Prompt de comando rodando F6(0,0)}, language={Matlab}]{code/prompt1.m}

\begin{flushleft}
Após essa análise inicial, voltamos a nossa função \emph{F6},
mas levando em consideração que esta possui uma quantidade
elevada de picos e vales no $\mathbb{R}^3$ necessitando de pelo
menos três abordagens para plotar essa função com os recursos
presentes no MATLAB\cite{MATLAB:2020b}.

%@online{WolframAlpha,
%@manual{MATLAB:2020b,
%

Mas antes de começar a análise com o matlab jogamos a função no
WolframAlpha\cite{WolframAlpha} para entender o que nós espera.
\end{flushleft}

\begin{figure}[!h]
	\begin{floatrow}
		\ffigbox{\includegraphics[scale = 0.8]{images/wolfram2.jpeg}}{
			\caption{Equação}
			\label{fig_wolfram:a}}

		\ffigbox{\includegraphics[scale = 0.8]{images/wolfram1.jpeg}}{
			\caption{WolframAlpha\cite{WolframAlpha}}
			\label{fig_wolfram:b}}
	\end{floatrow}
\end{figure}

\newpage
\subsection*{Primeira abordagem: plotando pontos em intervalos discretos}

Definimos x e y como arrays que definem a faixa de intervalo
$[-100..100]$ ao passo de $.5$ por exemplo.

\lstinputlisting[style=Matlab-editor,label={plot1},caption={Plot 1}, language={Matlab}]{code/myplot1.m}

\begin{figure}[h!]
\centering
\includegraphics[scale=.3]{images/myplot1.pdf}
\caption{Plot 1}
\label{fig:plot1_f6}
\end{figure}

\newpage
\begin{figure}[h!]
\centering
\includegraphics[scale=.3]{images/myplot1b.pdf}
\caption{Plot 1 visualizando os 3 eixos}
\label{fig:plot1b_f6}
\end{figure}

\newpage
\subsection*{Segunda abordagem: plotando pontos em intervalos discretos com redução do intervalo e um aumento substancial de pontos}

\begin{flushleft}
	Definimos x e y como arrays que definem a faixa de intervalo
	$[-1..1]$ ao passo de $.00005$ por exemplo.
\end{flushleft}

\lstinputlisting[style=Matlab-editor,label={plot2},caption={Plot 2}, language={Matlab}]{code/myplot2.m}

\newpage
\begin{figure}[h!]
\centering
\includegraphics[scale=.5]{images/myplot2.pdf}
\caption{Plot 2}
\label{fig:plot2_f6}
\end{figure}

\begin{flushleft}
	Definitivamente essa metodologia não diz muito sobre os picos e
	vales de nossa função \emph{F6}, os unicos pontos positívos até
	o momento são a observação da sua faixa de relevância e que o
	ponto de máximo se encontra na vizinhança do ponto $(0,0,z)$.
\end{flushleft}

\newpage
\subsection*{Terceira abordagem: plotando a função como uma curva de superfície em formato de malha em intervalos discretos com redução do intervalo de relevância}

\begin{flushleft}
	Mantemos o intervalo de relevância entre $[-100,100]$, todavia
	mantendo o número de pontos em torno de 1000 pontos.
\end{flushleft}

$x = linspace(-100, 100, 1000);$

\lstinputlisting[style=Matlab-editor,label={plot3},caption={Plot 3}, language={Matlab}]{code/myplot3.m}

\newpage
\begin{flushleft}
	Mesmo com essa nova forma de calcular os pontos e conectar os
	pequenos planos, ``mesh'' ou grades, não ficou claro conforme é
	visto no gráfico abaixo os máximos e mínimos de \emph{F6}.
\end{flushleft}

\begin{figure}[h!]
\centering
\includegraphics[scale=.5]{images/myplot3.pdf}
\caption{$X,Y \in [-100, 100]$}
\label{fig:plot3_f6}
\end{figure}

\begin{flushleft}
	Nesse ponto reduzimos o intervalo de relevância, o domínio,
	de forma drástica para $[-1,1]$ e mantemos o número de
	pontos em torno de 1000 pontos.
\end{flushleft}

$x = linspace(-1, 1, 1000);$

\lstinputlisting[style=Matlab-editor,label={plot4},caption={Plot 4}, language={Matlab}]{code/myplot4.m}

\newpage
\begin{flushleft}
	Mesmo seguindo essas dicas, não obtivemos sucesso na análise
	gráfica.
\end{flushleft}

\begin{figure}[h!]
\centering
\includegraphics[scale=.5]{images/myplot4.pdf}
\caption{$X,Y \in [-1, 1]$}
\label{fig:plot4_f6}
\end{figure}

\begin{flushleft}
	Sendo menos preciosista e após vários testes chegamos a uma
	possível interpretação da curva de superfície com o algorítmos
	subsequente. Para tanto alteramos o domíneo até o ponto desse
	ser mais substancialmente indicativo da estrutura espacial de
	\emph{F6}.
\end{flushleft}

\lstinputlisting[style=Matlab-editor,label={plot5},caption={Plot 5}, language={Matlab}]{code/myplot5.m}

\newpage
\begin{figure}[H]
	\begin{floatrow}
		\raggedleft
		\makebox[0pt]{
			\ffigbox{
				\includegraphics[scale = 0.4]{images/myplot5a.pdf}}{
				\caption{$X,Y \in [-5, 5]$}
				\label{fig:plot5a_f6}}

			\ffigbox{\includegraphics[scale = 0.4]{images/myplot5b.pdf}}{
				\caption{$X,Y \in [-10, 10]$}
				\label{fig:plot5b_f6}}
		}
	\end{floatrow}
\end{figure}

\newpage
\begin{flushleft}
	\textcolor{black}{
		``
		Sendo assim, podemos afirmar que \emph{F6} possui vários
		máximos locais. Formalmente, um ponto máximo local é um
		ponto no espaço de entrada tal que todas as outras entradas
		em uma pequena região perto desse ponto produzem valores
		menores quando aplicadas na função multivariável.
		''
	}
\end{flushleft}

\begin{figure}[h!]
\centering
\includegraphics[scale=.5]{images/myplot5c.pdf}
\caption{$X,Y \in [-20, 20]$}
\label{fig:plot5c_f6}
\end{figure}

\newpage
\section*{b)}

\begin{flushleft}
	Baseando-se no exemplo fornecido no gaot (binaryExample.m),
	implemente um programa baseado em GA (GA1) para maximizar a
	função f6 com as seguintes características:

	\begin{itemize}
		\item representação binária (binário codificando real);
		\item domínio de x e y $[-100, 100]$;
		\item epsilon = $1e-06$ (precisão);
		\item aptidão é a avaliação;
		\item População = $100$;
		\item Total de indivíduos = $40.000$;
		\item Taxa de Crossover = 80\%;
		\item Taxa de Mutação = 1\%;
		\item função de seleção (\emph{roulette weel})
	\end{itemize}

	Antes de mais nada, precisamos instalar a biblioteca no
	MATLAB\cite{MATLAB:2020b}, no nosso caso consiste em
	adicionar a pasta no caminho de leitura do programa,
	conforme mostrado abaixo.
\end{flushleft}

\lstinputlisting[style=Matlab-editor,label={prompt2},caption={Instalando biblioteca gaot}, language={Matlab}]{code/prompt2.m}

\begin{flushleft}
	Reescrevemos a função \emph{F6} para que ela seja compatível
	com a biblioteca \emph{GAOT}.
\end{flushleft}

\lstinputlisting[style=Matlab-editor,label={NewF6},caption={Reimplementação de F6}, language={Matlab}]{code/l2/F6.m}

\begin{flushleft}
	Implementamos abaixo, o maximizador de \emph{F6} escrevendo
	o programa ``maximizaF6.m'' seguindo o modelo apresentado no
	exemplo ``binaryExample.m''.
\end{flushleft}

\lstinputlisting[style=Matlab-editor,label={prog1},caption={Realiza Maximização da função \emph{F6}}, language={Matlab}]{code/l2/maximizaF6.m}

\begin{flushleft}
	Após rodar o programa teremos a seguinte saída:
\end{flushleft}

\begin{equation}
\begin{split}
	x &= -3.7253e-07 = -3.7253*10^{-7} = -0.0000003725 \\
	y &= -2.0675e-04 = -2.0675*10^{-4} = 0.0002067500 \\
	z &= 1.000000
\end{split}
\end{equation}

\lstinputlisting[label={output},linerange={33-38},caption={Saida do programa \emph{maximizaF6.m}}, language={Matlab}]{code/l2/maximizaF6.out}

\begin{flushleft}
	Nossa resposta corresponde exatamente a proposição inicial,
	onde x e y estão no entrorno de $(0,0)$ e $z = 1$. Conforme
	verificamos mais de uma vez ao efetuarmos $F6(0,0)$ e
	confirmar que tem imagem de valor 1.
\end{flushleft}

\newpage
\section*{b.1)}
\begin{flushleft}
	Quantos bits tem o cromossomo? Explique.

	Utilizando o programa \emph{maximizaF6.m} obtemos um total de 28 bits.
	Um binário também pode representar um número real $X_{R}$
	$[X_{min}X_{máx}]$, com precisão de p casas decimais. Para isso são
	necessários K bits, sendo K calculado pela inequação:
	\begin{equation}
	\begin{split}
		2^{K} &\leq (X_{max}-X_{min})\times 10^{p}\\
		\Rightarrow & log (2^{K}) \leq log ((X_{max}-X_{min})\times 10^{p}) \\
		\Rightarrow & log (2^{K}) \leq log ((100-(-100))\times 10^{6}) \\
		\Rightarrow & log (2^{K}) \leq log (200\times 10^{6}) \\
		\Rightarrow & K \times log 2 \leq log (2\times 10^{8}) \\
		\Rightarrow & K \times log 2 \leq log 2 + log 10^{8} \\
		\Rightarrow & K \times log 2 \leq log 2 + 8\times log 10 \\
		\Rightarrow & K \times log 2 \leq log 2 + 8 \\
		\Rightarrow & K \leq 1 + \frac{8}{log 2} \\
		\Rightarrow & K \leq 1 + \frac{8}{0.301} \\
		\Rightarrow & K \leq 1 + 26.6 \\
		\Rightarrow & K \leq 27.6 \\
		\Rightarrow & K \leq 28 \\
		\Rightarrow & K = 28
	\end{split}
	\end{equation}

\end{flushleft}


\lstinputlisting[style=Matlab-editor,label={bits},firstnumber=69,linerange={69-81}, caption={Função que obtem o número de bits em \emph{``maximizaF6.m''}}, language={Matlab}]{code/l2/maximizaF6b.m}

\lstinputlisting[label={outputb},linerange={33-42},caption={Saida do programa \emph{maximizaF6b.m}}, language={Matlab}]{code/l2/maximizaF6b.out}

\newpage
\section*{b.2)}

\begin{flushleft}
	Gere as curvas referentes ao melhor indivíduo de cada geração e
	à média dos indivíduos de cada geração. Comente os resultados
	obtidos.

	Um dos retornos da função \emph{ga} é a variável ``trace''
	que consiste numa matriz de quatro colunas, onde a primeira
	coluna é o número da geração por linha, para uma geração
	igual a 200 teremos uma coluna com valores de 1 a 200, a
	segunda coluna contém o melhor avaliado a cada
	geração e a terceira coluna a média dos indivíduos por
	geração.

\end{flushleft}


\lstinputlisting[style=Matlab-editor,label={bits},firstnumber=69,linerange={69-73}, caption={Programa que obtem os gráficos de melhor indivíduo e média por geração em \emph{``maximizaF6b2.m''}}, language={Matlab}]{code/l2/maximizaF6b2.m}

\newpage
\begin{figure}[h!]
\centering
\includegraphics[scale=.6]{images/melhorIndividuoACadaGeracao.pdf}
\caption{Melhor indivíduo de cada geração}
\label{fig:melhorIndividuoACadaGeracao}
\end{figure}

\newpage
\begin{figure}[h!]
\centering
\includegraphics[scale=.6]{images/mediaDosIndividuosACadaGeracao.pdf}
\caption{Média dos indivíduos de cada geração}
\label{fig:mediaDosIndividuosACadaGeracao}
\end{figure}

\newpage
\section*{c)}

\begin{flushleft}
	Repita o item (b.2) (Programa GA2) para normalized geometric
	select (q=0,20). Comente os resultados obtidos.
\end{flushleft}

\lstinputlisting[style=Matlab-editor,label={bits},firstnumber=39,linerange={39-45}, caption={Programa que obtem os gráficos de melhor indivíduo e média por geração em \emph{``maximizaF6c.m''}}, language={Matlab}]{code/l2/maximizaF6c.m}

\newpage
\begin{figure}[h!]
\centering
\includegraphics[scale=.6]{images/melhorIndividuoACadaGeracaoC.pdf}
\caption{Melhor indivíduo de cada geração utilizando seleção por normalização geométrica}
\label{fig:melhorIndividuoACadaGeracaoC}
\end{figure}

\newpage
\begin{figure}[h!]
\centering
\includegraphics[scale=.6]{images/mediaDosIndividuosACadaGeracaoC.pdf}
\caption{Média dos indivíduos de cada geração utilizando seleção por normalização geométrica}
\label{fig:mediaDosIndividuosACadaGeracaoC}
\end{figure}

\begin{flushleft}
	Ao utilizando a seleção por normalização geométrica conseguimos
	um ganho superior pois a média é maior por geração e atingimos a
	maximização mais rapidamente.
\end{flushleft}

\bibliographystyle{plain}
\bibliography{references}

\end{document}
