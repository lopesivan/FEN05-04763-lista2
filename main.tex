\documentclass{article}
\usepackage[utf8]{inputenc}

\title{%
  Tópicos Especiais em Eletrônica\\
  \large Lista 2\\
    Resolução}

\author{Ivan Carlos}
\date{October 2021}

\usepackage{natbib}
\usepackage{amsmath}  % matemática
\usepackage{amssymb}

\usepackage{blindtext} % links
\usepackage{hyperref}  % links

\usepackage{graphicx}
\usepackage{geometry}

\usepackage[utf8]{inputenc}
\usepackage[T1]{fontenc}
\usepackage[brazil]{babel}

\usepackage{color}    % suporte a cor
\usepackage{xcolor}
\usepackage{listings}
\usepackage{courier}
\lstset{
	basicstyle=\footnotesize\ttfamily, % Standardschrift
	%numbers=left,                % Ort der Zeilennummern
	numberstyle=\tiny,            % Stil der Zeilennummern
	%stepnumber=2,                % Abstand zwischen den Zeilennummern
	numbersep=5pt,                % Abstand der Nummern zum Text
	tabsize=2,                    % Groesse von Tabs
	extendedchars=true,           %
	breaklines=true,              % Zeilen werden Umgebrochen
	keywordstyle=\color{gray},
	%frame=b,
	%        keywordstyle=[1]\textbf,    % Stil der Keywords
	%        keywordstyle=[2]\textbf,    %
	%        keywordstyle=[3]\textbf,    %
	%        keywordstyle=[4]\textbf,   \sqrt{\sqrt{}} %
	showspaces=false,           % Leerzeichen anzeigen ?
	showtabs=false,             % Tabs anzeigen ?
	xleftmargin=17pt,
	framexleftmargin=17pt,
	framexrightmargin=5pt,
	framexbottommargin=4pt,
	%backgroundcolor=\color{lightgray},
	showstringspaces=false   % Leerzeichen in Strings anzeigen ?
	literate=
	{á}{{\'a}}1 {é}{{\'e}}1 {í}{{\'i}}1 {ó}{{\'o}}1 {ú}{{\'u}}1
	{Á}{{\'A}}1 {É}{{\'E}}1 {Í}{{\'I}}1 {Ó}{{\'O}}1 {Ú}{{\'U}}1
	{à}{{\`a}}1 {è}{{\`e}}1 {ì}{{\`i}}1 {ò}{{\`o}}1 {ù}{{\`u}}1
	{À}{{\`A}}1 {È}{{\'E}}1 {Ì}{{\`I}}1 {Ò}{{\`O}}1 {Ù}{{\`U}}1
	{ä}{{\"a}}1 {ë}{{\"e}}1 {ï}{{\"i}}1 {ö}{{\"o}}1 {ü}{{\"u}}1
	{Ä}{{\"A}}1 {Ë}{{\"E}}1 {Ï}{{\"I}}1 {Ö}{{\"O}}1 {Ü}{{\"U}}1
	{â}{{\^a}}1 {ê}{{\^e}}1 {î}{{\^i}}1 {ô}{{\^o}}1 {û}{{\^u}}1
	{Â}{{\^A}}1 {Ê}{{\^E}}1 {Î}{{\^I}}1 {Ô}{{\^O}}1 {Û}{{\^U}}1
	{ã}{{\~a}}1 {ẽ}{{\~e}}1 {ĩ}{{\~i}}1 {õ}{{\~o}}1 {ũ}{{\~u}}1
	{Ã}{{\~A}}1 {Ẽ}{{\~E}}1 {Ĩ}{{\~I}}1 {Õ}{{\~O}}1 {Ũ}{{\~U}}1
	{œ}{{\oe}}1 {Œ}{{\OE}}1 {æ}{{\ae}}1 {Æ}{{\AE}}1 {ß}{{\ss}}1
	{ű}{{\H{u}}}1 {Ű}{{\H{U}}}1 {ő}{{\H{o}}}1 {Ő}{{\H{O}}}1
	{ç}{{\c c}}1 {Ç}{{\c C}}1 {ø}{{\o}}1 {å}{{\r a}}1 {Å}{{\r A}}1
	{€}{{\euro}}1 {£}{{\pounds}}1 {«}{{\guillemotleft}}1
	{»}{{\guillemotright}}1 {ñ}{{\~n}}1 {Ñ}{{\~N}}1 {¿}{{?`}}1 {¡}{{!`}}1
}

\usepackage{caption}
\input{etc/color.tex}

%\usepackage{auto-pst-pdf} % ps

\usepackage{epstopdf}
\usepackage{pstricks} 
\usepackage{pst-plot}
\usepackage{pstricks-add}
\usepackage{epsf}


%\geometry{left=2.5cm,right=2.5cm,top=2.5cm,bottom=2.5cm}

% mudando o idioma em listing -> Listagem
\renewcommand{\lstlistingname}{Listagem}
%\renewcommand{\lstlistingname}{Algorithm}
%\renewcommand{\lstlistlistingname}{Lista de Códigos Fonte}

% desabilita seções numeradas
\setcounter{secnumdepth}{0} % Do not enumerate a section

\begin{document}

\maketitle

\section{a)}

O primeiro passo é escrever nossa função abaixo.

\begin{equation} \label{eq1}
    F6(x,y) = 0.5-\frac{(\sin\sqrt{x^2+y^2})^2 -0.5}{(1+(0.0001)\cdot (x^2+y^2)))^2}
\end{equation}

No Matlab como uma abordagem inicial, escrevemos a função lembrando que as entradas e saídas são vetoriais. Não é uma função difícil de ser expressa, mas devemos tomar cuidado com a operações ponto a ponto (``.*'', ``./''). 

\begin{figure}[h!]
\centering
\includegraphics[scale=.3]{fun.png}
\caption{Modelo da função F6}
\label{fig:func_f6_model}
\end{figure}

Dentro do diretório de trabalho criamos o arquivo ``F6.m'', conforme mostrado abaixo.

\lstinputlisting[label={F6},caption={Arquivo F6.m}, language={Matlab}]{code/F6.m}

Para invocar a função basicamente digitamos \emph{F6} munido de suas entradas.

Dessa forma testamos as entradas $X=0, Y=0$ mencionadas no item e verificamos se a saída é $Z=1$.

\lstinputlisting[label={prompt1},caption={Prompt de comando rodando F6(0,0)}, language={Matlab}]{code/prompt1.m}

Após essa análise inicial, voltamos a nossa função \emph{F6}, mas levando em consideração que esta possui uma quantidade elevada de picos e vales no $\mathbb{R}^3$ necessitando de pelo menos três abordagens para plotar essa função com os recursos presentes no matlab.

\subsection*{Primeira abordagem: plotando pontos em intervalos discretos}

Definimos x e y como arrays que definem a faixa de intervalo $[-100..100]$ ao passo de $.5$ por exemplo.


\lstinputlisting[label={plot1},caption={Plot 1}, language={Matlab}]{code/myplot1.m}

\newpage
\begin{figure}[h!]
\centering
\includegraphics[scale=.5]{myplot1.pdf}
\caption{Plot 1}
\label{fig:plot1_f6}
\end{figure}

\newpage
\begin{figure}[h!]
\centering
\includegraphics[scale=.5]{myplot1b.pdf}
\caption{Plot 1 visualizando os 3 eixos}
\label{fig:plot1b_f6}
\end{figure}


\subsection*{Segunda abordagem: plotando pontos em intervalos discretos com redução do intervalo e um aumento substancial de pontos}

Definimos x e y como arrays que definem a faixa de intervalo $[-1..1]$ ao passo de $.00005$ por exemplo.


\lstinputlisting[label={plot2},caption={Plot 2}, language={Matlab}]{code/myplot2.m}

\newpage
\begin{figure}[h!]
\centering
\includegraphics[scale=.5]{myplot2.pdf}
\caption{Plot 2}
\label{fig:plot2_f6}
\end{figure}

Definitivamente essa metodologia não diz muito sobre os picos e vales de nossa função \emph{F6}, os unicos pontos positívos até o momento é a observação da sua faixa de relevância e que o ponto de máximo se encontra na vizinhança do ponto $(0,0,z)$.

\subsection*{Terceira abordagem: plotando a função como uma curva de superfície em intervalos discretos com redução do intervalo de relevância}

%Definimos x e y como arrays que definem a faixa de intervalo $[-1..1]$ ao passo de $.00005$ por exemplo.



\bibliographystyle{plain}
\bibliography{references}

\end{document}
